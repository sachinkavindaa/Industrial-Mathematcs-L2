\chapter{Conclusion}
\label{chap:06}

\paragraph{}

The goal of this project is to provide insight for the design of coupled one dimensions (1D) and two dimensions (2D) spring mass models. During the project, the motion of the coupled oscillator systems (both 1D and 2D) has been studied  and observed numerically using the Runge Kutta Fourth order method. Throughout the project, there are three objectives have been considered. Investigate the motion for undamped linear spring mass systems with coupled two degrees of freedom, investigate the motion for undamped linear spring mass systems with coupled many degrees of freedom and investigate the two-dimensional spring motion: the mass m is free to move in the XY plane were discussed graphically and numerically.

First, the motion for undamped linear spring systems with coupled two degrees of freedom was analyzed to provide insight for the its numerical behaviour. In this case, analytical formulas for the approximate eigenvectors and eigenvalues were developed. The changing the parameters values (mass of particle, initial displacement and etc )
procedure which makes it relatively simple to perform based on the approximate eigenvalues and eigenvectors. Although, different behaviours. In this case there are three cases were discussed. First, started all the two masses and spring constant are different with each other. But their initial displacements are 0.005 $m$ both $m_1$ and $m_2$ masses respectively. After the system was released, the displacement and speed of the system corresponded to each other. That because which has same initial displacements. After that, the initial displacements of different values were different from each other and the other parameters remained constant. It could be seen that the anti symmetrical behaviour of the figures. Finally, changed the only initial displacement of $x_1$ was changed while $x_2$ was zero.  Other parameters are kept constant. It is interesting to note that
the system demonstrates a beating phenomenon, in which energy is transferred cyclically from one mass to the next. When the two eigen frequencies are approximately equal, this common behaviour basically happens. Interestingly, it could be seen that  mass ratio ($\mu = \frac{m_2}{m_1}$) should be $\mu$ $\geq$ 0 if there is no energy leakage. 

Then, the motion for undamped linear spring mass systems with coupled many degrees of freedom was observed. In this case, the application of dynamics of one dimensional (1D) infinite monoatomic chain of atoms were  yielded with different conditions. In this scenario, frequency ($\omega$), phase velocity ($V_P$) and group velocity ($V_G$) played a major throughout the experiment. In the reuslts, dispersion curve clearly shows that for one value of $\omega$ there are several values of wave-vector $K_p$.  

When discussing simple-harmonic motion in two dimensional, the motion can always be divided into two components, each directed along one of the two coordinate axes ($X$ and $Y$ axes). Lastly, two dimensional sprig mass model was explained with figures. There are four case studies (case study 01. case study 01, case study 03 and case study 04) were discussed considering 2D motion the connector points where one end of every spring connects are considered fixed in space and in line with the two-dimensional plane’s $X$ and$Y$ axes. Case study 01 reveled numerical results which included only the mass initial displacement towards the positive $X$ axis. It can be seen that system oscillates $X$ component (displacement and velocity) if there is an initial displacement towards the positive $X$ axis. In the case study 02, If both initial displacement defined with value it can be seen that the displacement and velocity of mass along both $X$ and $Y$ directions. Behaviour of particle depends on the input initial value. Case study 03, isotropic oscillations of mass $m$ moving in the two-dimensional will be discussed. The results of case study 03 have been proved  that they will differ according to their relevant phase difference $\Delta$. Corresponding phase difference are  $\Delta = 0 $, $\Delta = \pi/4 $ , $\Delta = \pi/2 $ and $\Delta = 3\pi/4 $. Note that when ${\mit\Delta}=0$ the trajectory generates into a straight-line  and for the phase difference values it can be seen that system has an ellipse. Concluded, the easiest way to examine this type of motion is by actually looking at figures of derived, the trajectory carried out by the particle. Looking at the case study 04, the dynamics of a mass coupled springs are depicted by the anisotropic oscillator model. In this process, the angular frequency of the motion along the $X$ axis may be different from the angular frequency of the motion along the $y$ axis.  But other initial conditions keep remain as discussed in earlier. The trajectory described by the particle is called a Lissajous curve when $\omega_x = 2\omega_y$. Although when $\omega_x = \sqrt{2}\omega_y$ output trajectory described by the particle is called a quasi periodic. 

Many engineering problems are modelled and solved using mass-spring systems as the physical foundation. Such models are utilised in the construction of architectural structures or, for example, in the development of sportswear, among other things in modern life. Of fact, in real life, the system of equations of spring masses can be far more complex.