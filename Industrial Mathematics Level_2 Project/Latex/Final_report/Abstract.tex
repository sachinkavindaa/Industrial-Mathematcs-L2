\chapter*{Abstract}
\paragraph{}

The vibration phenomenon can be noticed in many common settings, including speaking, vision, auditioning, and other activities involving human relationships due to mechanical waves. This phenomena is also required in various engineering applications related to our daily lives, where vibration is exploited to increase the comfort of various human activities. Because of their complexities, vibrations are difficult to characterize and estimate quantitatively. Despite the development of several mathematical models, it remains challenging to balance computational complexity and estimate the accuracy without a numerical analysis. This project focuses on adding physical accuracy and visual authenticity to existing one dimensional (1D) and two dimensional (2D) mass-spring models based on numerical analysis. The Lagrangian approach was used to obtain the motion of connected oscillator systems in both 1D and 2D. MATLAB demonstrates that the numerical process of the Runge Kutta fourth-order method is more computationally efficient. Firstly, the 1D mass-spring model parameters for the two degree of freedom (2DOF) are determined using various conditions. Interestingly, with the same mass and spring stiffness, it is demonstrated that the 1D mass-spring model has accurate resistance to lateral displacement characteristics, which is one of the most vital aspects of the accuracy of the 1D mass-spring model. Consequently, in the case of the imaginary frequency, there would be energy leakage, which defies the system's energy conservation criterion. In fact, the mass ratio $\mu$ should be greater than zero. The procedure is then extended to the 2D mass-spring model, where the mass $m$ could indeed move freely in the $XY$ plane. Isotope oscillations could be noticed in the 2D mass-spring model with varying phase difference values ($\Delta = 0 , \pi/4 , \pi/2$  and $3\pi/4 $). Since this angular frequency of oscillation in the $X$-direction is equal to the angular frequency of motion in the $Y$-direction, the particle's path would be a closed trajectory. The particle observed in the trajectory is called an anisotropic oscillation when $\omega_x = 2\omega_y$ and $\omega_x = \sqrt{2}\omega_y$. The accuracy and behaviour of mass-spring models were tested numerically under various scenarios to illustrate the efficacy of our technique.





































































































































































































































































