\chapter*{Appendix}


\textbf{\Large Code 01}\\

\begin{framed}
\begin{verbatim}
    clc;
clear all;

%Degree of freedom


%input mass_1
M1 = 0.815
%input mass_2
M2 = 3.099

%Inputs for spring constants.
K1 = 41.5

%Inputs for spring constants.
K2 = 138

%Inputs for spring constants.
K3 = 157.8

%Mass matrix
M = [M1 0; 0 M2];

%Stiffness matrix
K = [K1+K2 -K2 ; -K2 K2+K3];

%Estimation natural frequncies and mode shapes

[modeShape fr] = eig(K,M);

%Coffecient matrix

A00 = zeros(2);
A11 = eye(2);
CC = [A00 A11; -inv(M)*K A00];
global CC

%%Time step and time vector

max_freq = max(sqrt(diag(fr))/(2*pi)); %maximum frequency in Hz
dt = 1/(max_freq*20);
Nstep = 100;
time = 0:dt:Nstep*dt;

y0 = [0.005 0.005 0 0]; %[disp1 disp2 vol1 vol2] inatial condition
[tsol,ysol] = ode45('testcode',time,y0);

y1 = ysol(:,1);
y2 = ysol(:,2);
v1 = ysol(:,3);
v2 = ysol(:,4);

idxmin = find(y1 == max(y1))
idxmax = find(y1 == min(y1))



figure()
plot(time,y1,'--r','linewidth',1.5);
hold on
plot(time,y2,'-.b','linewidth',1.5);
hold off
% plot(y1(idxmin),y1(idxmax),'*','MarkerFaceColor','red',...
%     'MarkerSize',10)
% % hold on
% % plot(y2(idxmax),y1(idxmax),'-p','MarkerFaceColor','black',...
% %     'MarkerSize',10)
% hold off
xlabel('Time (s)');
ylabel('Displacement (m)');
title('Displacement VS Time')
legend('Mass 1','Mass 2');
grid on

figure()
plot(time,v1,'--r','linewidth',1.5);
hold on
plot(time,v2,'-.b','linewidth',1.5);
hold on
xlabel('Time(s)');
ylabel('Velocity (m/s)');
title('Velocity VS Time')
legend('Mass 1','Mass 2');
grid on

figure()
plot(y1,y2,'--r','linewidth',1);
hold on
xlabel('Time(s)');
ylabel('Velocity (m/s)');
title('Velocity VS Time')
legend('Mass 1','Mass 2');
grid on


\end{verbatim}
\end{framed}

\newpage

\begin{framed}
\begin{verbatim}
    clear all; 
close all;
clc;
tic

%input mass_1
m1 = input('Enter value of mass_1 (kg):   \n');
%input mass_2
m2 = input('Enter value of mass_2 (kg):   \n');

%Inputs for spring constants.
k1 = input('Enter value of the spring constants (N/m): \n');
%Inputs for spring constants.
k2 = input('Enter value of the spring constants (N/m): \n');
%Inputs for spring constants.
k3 = input('Enter value of the spring constants (N/m): \n');

x10=1; % mass 1 initial position 
x20=1; % mass 2 initial position 

v10=0; % mass 1 initial velocity 
v20=0; % mass 2 initial velocity 

% Set time step 
simTime=1; % simulation time 
tStep=0.01; % simulation time step

iterations=simTime/tStep;
t=0:iterations;

% variables for speed
x1=zeros(iterations,1);
x1(1,:)=x10;
x2=zeros(iterations,1);
x2(1,:)=x20;

v1=zeros(iterations,1);
v1(1,:)=v10;
v2=zeros(iterations,1);
v2(1,:)=v20;

a1=zeros(iterations,1);
a1(1,:)=( k3*(x20-x10) + -k1*x10)/m1;
a2=zeros(iterations,1);
a2(1,:)= (-(k2+k3)*x20 + k3*x10)/m2;

% Set up video
v=VideoWriter('twomass.avi');
v.FrameRate=100;
open(v);

set(gcf, 'Position',  [50, 50, 700, 600])%positions of plot

for n=2:(iterations+1)
    
  %euler method
    
  x1(n,:)=x1(n-1,:)+v1(n-1,:)*tStep;
  x2(n,:)=x2(n-1,:)+v2(n-1,:)*tStep;
  
  v1(n,:)=v1(n-1,:)+a1(n-1,:)*tStep;
  v2(n,:)=v2(n-1,:)+a2(n-1,:)*tStep;
  
  a1(n,:)=(k3*((x2(n,:))-(x1(n,:)))+-k1*(x1(n,:)))/m1;
  a2(n,:)=(-(k2+k3)*(x2(n,:))+k3*(x1(n,:)))/m2;

  % Plot for the video
  subplot(4,1,1)
  
  hold on;
  plot(t(:,1:n),x1(1:n,:),'r')
  plot(t(:,1:n),x2(1:n,:),'m')
  
  xlim([0 iterations])
  ylabel('Position(m)')
  legend('mass 1','mass 2')
  
  subplot(4,1,2)
  
  hold on;
  plot(t(:,1:n),v1(1:n,:),'b')
  plot(t(:,1:n),v2(1:n,:),'c')
 
  xlim([0 iterations])
  ylabel('Velocity(m/s)')
  legend('mass 1','mass 2')
  
  subplot(4,1,3)
  
  hold on;
  plot(t(:,1:n),a1(1:n,:),'g')
  plot(t(:,1:n),a2(1:n,:),'y')
 
  xlim([0 iterations])
  ylabel('Acceleration(m/s^2)')
  legend('mass 1','mass 2')
  
  subplot(4,1,4)
  
  hold on;
  fill([0 9 9 0],[0 0 1.5 1.5],'w'); % clears background
  
  plot([x1(n,:)+3 0],[0.25 0.25],'k'); % springs
  plot([x1(n,:)+3 x2(n,:)+6],[0.25 0.25],'k');
  plot([x2(n,:)+6 9],[0.25 0.25],'k');
  
  fill([x1(n,:)+2.75 x1(n,:)+3.25 x1(n,:)+3.25 x1(n,:)+2.75],
  [0 0 0.5 0.5],'g'); % mass 1
  fill([x2(n,:)+5.75 x2(n,:)+6.25 x2(n,:)+6.25 x2(n,:)+5.75],
  [0 0 0.5 0.5],'g'); % mass 2
  
  fill([0 0.1 0.1 0],[0 0 1 1],[0.5 0.5 0.5]);
  fill([8.9 9 9 8.9],[0 0 1 1],[0.5 0.5 0.5]);

  xlim([0 9]);
  ylim([0 1.5]);
  frame=getframe(gcf);
  writeVideo(v,frame)
  
end
close(v);

figure,

subplot(3,1,1)
hold on;
plot(t',x1,'r')
plot(t',x2,'m')
ylabel('Position (m)')
title('Position, Velocity, & Acceleration as a Function of Time')
legend('mass1','mass2')
grid on

subplot(3,1,2)
hold on;
plot(t',v1,'b')
plot(t',v2,'c')
ylabel('Velocity (m/s)')
legend('mass1','mass2')
grid on

subplot(3,1,3)
hold on;
plot(t',a1,'g')
plot(t',a2,'y')
ylabel('Acceleration (m/s^2)')
xlabel('time (iterations)')
legend('mass1','mass2')
grid on

toc


\end{verbatim}
\end{framed}

\begin{framed}
\begin{verbatim}
    function [x_new,y_new] = sp1(x0,y0,x1,y1,n,r)
x=x0:((x1-x0)/n):x1;
y=((y1-y0)/(x1-x0))*x+y0-((y1-y0)/(x1-x0))*x0;
m=(y1-y0)/(x1-x0);
T=abs(atan(m));
s=length(x);
x_new=zeros(1,s);
y_new=zeros(1,s);
for i=1:s
  x_new(i)=x(i)+r*(-1)^(i)*sin(T);
  y_new(i)=y(i)+r*(-1)^(i)*cos(T);
end
\end{verbatim}
\end{framed}

\newpage

\textbf{\Large Code 02}\\






\begin{framed}
\begin{verbatim}

clear all; 
close all;
clc;

% Assume spring constants are equivalent and masses are equivelent.

%input Mass Value
m = input('Enter value of Mass Weight (kg):   \n');
%Input for spring constants.
k = input('Enter value of the spring constants (N/m): \n');
%Input for spring constants.
n = input('Number of Masses: \n');
% simulation time
t = input('Simulation time period: \n');

tStep=0.001;
mode=2; % must be <= n

% Create mass and stiffness matrices based on generalized FBD's

% mass matrix is an identity matrix
M=m.*eye(n); 
e=ones(n,1);

% stiffness matrix
K=spdiags([-k.*e 2*k.*e -k.*e],-1:1,n,n);

% make the K matrix into a full matrix
K=full(K); 

% Returns phi (eigenvectors) and lambda (eigenvalues) - M*phi*lambda=K*phi
[phi,lambda]=eig(K,M);
% Get natural frequencies from eigenvalues
Wn=sqrt(diag(lambda)); 

% Time step setup
% total number of iterations
iterations=t/tStep;

% time vector (for plotting purposes)
tVector=tStep.*(1:iterations); 

% Pre-allocate arrays for speed
x=zeros(iterations,n);
v=zeros(iterations,n);
a=zeros(iterations,n);

% initial displacements governed by desired mode
x(1,:)=phi(:,mode); 

% Iteratively solve equations of motion using Semi-implicit Euler
for i=2:iterations
    for j=1:n
        a(i,j)=(-K(j,:)./M(j,j))*(x(i-1,:)');
        v(i,j)=v(i-1,j)+a(i,j)*tStep;
        x(i,j)=x(i-1,j)+v(i,j)*tStep;
    end
end

% Plot the results
for k=1:n
    subplot(3,1,1); hold on; grid on;
    plot(tVector,x(:,k))
    subplot(3,1,2); hold on; grid on;
    plot(tVector,v(:,k))
    subplot(3,1,3); hold on; grid on;
    plot(tVector,a(:,k))
end

subplot(3,1,1);
title('Position, Velocity, and Acceleration as a Function of Time');
ylabel('Position (m)');
grid on;

subplot(3,1,2);
ylabel('Velocity (m/s)');
subplot(3,1,3);
ylabel('Acceleration (m/s^2)');
xlabel('Time (s)');
grid on;



\end{verbatim}
\end{framed}

\begin{framed}

\begin{verbatim}
    clc;
clear all;
close all;

%%%  User input %%% 


% spring costants kS  
   kS = 41.5;

% mass of atoms or particles m 
   m = 0.815;

% The equilibrium separation distance between masses 
   d = 0.01;
   
% Number of mass N
   N = 100;

% Parameters for propagation constant k
   kMin = -3.9999; 
   kMax = 3.9999; 
   kN = 1501;

% Max angular frequency
   omega0 = sqrt(4*kS/m);

% Max GROUP velcoity   
   v0 = d*sqrt(kS/m);

% Propagation constant k [units: pi/d]   
   n = linspace(kMin,kMax,kN);
   k = n.*pi/d;

% Brillouin zone
   n1 = find(n>-1,1); 
   n2 = find(n>1,1);
   
% Wavelength   lambda  
   lambda = 2*pi./k;

% Angular frequency of propagating wave
   omega = omega0 .* abs( sin(n*pi/2) );

% Phase velocity  vP
   vP = abs( (omega)./(k) );

% Group Velocity
   vG = v0 .* abs( cos(n*pi/2) );

%%% MASS DISPLACEMENTS %%%


% X axis
  x = linspace(0,N*d,N);

% Wavelengths
  wL1 = 1.2*N*d;
  wL2 = d/2;

% Mass displacements
   e1 = sin(2*pi*x/wL1);
   e2 = sin(2*pi*x/wL2);  

% Single set of displacements
  Ls = N*d;
  es = sin(2*pi*x/Ls);

  xs = linspace(0,N*d,501);
  Ls1 = N*d;
  es1 = sin(2*pi*xs/Ls1);
  
  Ls2 = d;
  es2 = sin(2*pi*xs/Ls2);

  % GRAPHICS ============================================================ 

   figure(1)
   subplot(3,1,1)
   xP = n; 
   yP = omega;
   plot(xP,yP,'b','linewidth',2)
   hold on
   xP = n(n1:n2); 
   yP = omega(n1:n2);
   plot(xP,yP,'r','linewidth',3)
   
   grid on
   set(gca,'fontsize',12)
   xlabel('k   [ \pi  / d ]')
   ylabel('\omega')
   title(' Dispersion relationship \omega VS k')
   
   subplot(3,1,2)
   xP = n; 
   yP = vP;
   plot(xP,yP,'b','linewidth',2)
   hold on
   xP = n(n1:n2); 
   yP = vP(n1:n2);
   plot(xP,yP,'r','linewidth',3)
   
   grid on
   set(gca,'fontsize',12)
   xlabel('k   [ \pi  / d ]')
   ylabel('v_P')
   title(' Dispersion relationship v_P VS k')
   
   subplot(3,1,3)
   xP = n; 
   yP = vG;
   plot(xP,yP,'b','linewidth',2)
   hold on
   xP = n(n1:n2);
   yP = vG(n1:n2);
   plot(xP,yP,'r','linewidth',3)
   
   grid on
   set(gca,'fontsize',12)
   xlabel('k   [ \pi  / d ]')
   ylabel('v_G')
   title(' Dispersion relationship v_G VS k')
   
   figure(2)
   
 
   subplot(2,1,1)
   xP = x; yP = e1;
   hPlot = stem(xP,yP,'o');
   set(hPlot,'markersize',4,'markerfacecolor','b')
   grid on
   set(gca,'fontsize',12)
   ylabel('left   right')
   
   subplot(2,1,2)
   xP = x; yP = e2;
   hPlot = stem(xP,yP,'o');
   set(hPlot,'markersize',4,'markerfacecolor','b')
   grid on
   set(gca,'fontsize',12)
   xlabel('x')
   ylabel('left   right')

   suptitle('Displacements on the monatomic lattice')

   
 
\end{verbatim}
   \end{framed}        

\newpage

\textbf{\Large Code 03}\\

\begin{framed}
\begin{verbatim}
    clear all
close all
clc

% Initialise variables
%input Mass Value
m = 0.815
%Input for spring constants.
k = 41.5
%Input length of spring.
l = 0.01

h = 0.001;
t = 0:h:10;
N = length(t);

% Initialise vectors
z1    = zeros(N,1);
z2    = zeros(N,1);
z3    = zeros(N,1);
z4    = zeros(N,1);

w = (-k)/m;

% Initialise derivative functions

dz1 = @(t, z1, z2, z3, z4) z3;               % dx/dt = z1 
dz2 = @(t, z1, z2, z3, z4) z4;               % dy/dt = z2 
dz3 = @(t, z1, z2, z3, z4)
w*( (((z1+l)*(sqrt((z1+l).^2 + z2.^2)-l))./(sqrt( (z1+l).^2 + z2.^2 )))+ 

(( z1*(sqrt( (z2+l).^2 + z1.^2)-l))./(sqrt( (z2+l).^2 + z1.^2)))-

(( (l-z1)*(sqrt( (l-z1).^2 + z2.^2)-l))./(sqrt( (l-z1).^2 + z2.^2)))+

((z1*(sqrt((l-z2).^2 + z1.^2)-l))./(sqrt((l-z2).^2 + z1.^2 ))));

% dz1/dt = z3

dz4 = @(t, z1, z2, z3, z4) w*( ((z2*(sqrt((z1+l).^2 + z2.^2)-l))./

(sqrt((z1+l).^2 + z2.^2)))+((z2*(sqrt((l-z1).^2 + z2.^2)-l))./

(sqrt( (l-z1).^2 + z2.^2)))+(((z2+l)*(sqrt(z1.^2 + (z2+l).^2)-l))./

(sqrt(z1.^2 + (z2+l).^2)))+(((z2-l)*(sqrt(z1.^2 + (l-z2).^2)-l))./

(sqrt(z1.^2 + (l-z2).^2))));  % dz2/dt = z4 

%Starting conditions

z1(1) = 0.005; 
z2(1) = 0;
z3(1) = 0;
z4(1) = 0;

% Initialise K vectors

kz1 = zeros(1,4); % to store K values for z1
kz2 = zeros(1,4); % to store K values for z2
kz3 = zeros(1,4); % to store K values for z3
kz4 = zeros(1,4); % to store K values for z4

b = [1 2 2 1];   % RK4 coefficients


% Iterate, computing each K value in turn, then the i+1 step values by
Runge kutta method

for i = 1:(N-1)
    kz1(1) = dz1(t(i), z1(i), z2(i), z3(i) , z4(i));
    kz2(1) = dz2(t(i), z1(i), z2(i), z3(i) , z4(i));
    kz3(1) = dz3(t(i), z1(i), z2(i), z3(i) , z4(i));
    kz4(1) = dz4(t(i), z1(i), z2(i), z3(i) , z4(i));

    kz1(2) = dz1(t(i) + (h/2), z1(i) + (h/2)*kz1(1), z2(i) + (h/2)*kz2(1), 
    z3(i) + (h/2)*kz3(1) , z4(i) + (h/2)*kz4(1));
    kz2(2) = dz2(t(i) + (h/2), z1(i) + (h/2)*kz1(1), z2(i) + (h/2)*kz2(1), 
    z3(i) + (h/2)*kz3(1) , z4(i) + (h/2)*kz4(1));
    kz3(2) = dz3(t(i) + (h/2), z1(i) + (h/2)*kz1(1), z2(i) + (h/2)*kz2(1), 
    z3(i) + (h/2)*kz3(1) , z4(i) + (h/2)*kz4(1));
    kz4(2) = dz4(t(i) + (h/2), z1(i) + (h/2)*kz1(1), z2(i) + (h/2)*kz2(1), 
    z3(i) + (h/2)*kz3(1) , z4(i) + (h/2)*kz4(1));
    
    kz1(3) = dz1(t(i) + (h/2), z1(i) + (h/2)*kz1(2), z2(i) + (h/2)*kz2(2), 
    z3(i) + (h/2)*kz3(2) , z4(i) + (h/2)*kz4(2));
    kz2(3) = dz2(t(i) + (h/2), z1(i) + (h/2)*kz1(2), z2(i) + (h/2)*kz2(2), 
    z3(i) + (h/2)*kz3(2) , z4(i) + (h/2)*kz4(2));
    kz3(3) = dz3(t(i) + (h/2), z1(i) + (h/2)*kz1(2), z2(i) + (h/2)*kz2(2), 
    z3(i) + (h/2)*kz3(2) , z4(i) + (h/2)*kz4(2));
    kz4(3) = dz4(t(i) + (h/2), z1(i) + (h/2)*kz1(2), z2(i) + (h/2)*kz2(2), 
    z3(i) + (h/2)*kz3(2) , z4(i) + (h/2)*kz4(2));

    kz1(4) = dz1(t(i) + h, z1(i) + h*kz1(3), z2(i) + h*kz2(3), z3(i) +
    h*kz3(3) , z4(i)+ h*kz4(3));
    kz2(4) = dz2(t(i) + h, z1(i) + h*kz1(3), z2(i) + h*kz2(3), z3(i) + 
    h*kz3(3) , z4(i)+ h*kz4(3));
    kz3(4) = dz3(t(i) + h, z1(i) + h*kz1(3), z2(i) + h*kz2(3), z3(i) + 
    h*kz3(3) , z4(i)+ h*kz4(3));
    kz4(4) = dz4(t(i) + h, z1(i) + h*kz1(3), z2(i) + h*kz2(3), z3(i) +
    h*kz3(3) , z4(i)+ h*kz4(3));
    
    z1(i+1) = z1(i) + (h/6)*sum(b.*kz1);      
    z2(i+1) = z2(i) + (h/6)*sum(b.*kz2);      
    z3(i+1) = z3(i) + (h/6)*sum(b.*kz3);
    z4(i+1) = z4(i) + (h/6)*sum(b.*kz4);
     
    
end   

%displacments of springs

spring_1 = sqrt((z1+l).^2+ z2.^2)-l ; %l1
spring_4 = sqrt((z2+l).^2+ z1.^2)-l ; %l4
spring_2 = l- sqrt((l-z2).^2+ z1.^2); %l2
spring_3 = l- sqrt((l-z1).^2+ z2.^2); %l3

% Group together in one solution matrix

disp('Values of t z1 z2 z3 z4 spring_1 spring_2 spring_3 spring_4 ');
Answer_Matrix = [t' z1 z2 z3 z4 spring_1 spring_2 spring_3 spring_4 ];


%plot results

figure()
plot(t,z1,'--r','linewidth',1.5);
hold on
plot(t,z2,'-.b','linewidth',1.5);
hold on
xlabel('Time(s)');
ylabel('Displacement (m)');
title('Displacement VS Time')
legend('X direction', 'Y direction');
grid on

figure()
plot(t,z3,'--r','linewidth',1.5);
hold on
plot(t,z4,'-.b','linewidth',1.5);
hold on
xlabel('Time(s)');
ylabel('Velocity (m/s)');
title('Velocity VS Time')
legend('X direction', 'Y direction');
grid on

figure()
plot(t,spring_1,'linewidth',1);
hold on
plot(t,spring_2,'linewidth',1);
hold on
plot(t,spring_3,'linewidth',1);
hold on
plot(t,spring_4,'linewidth',1);
hold on
xlabel('Time(s)');
ylabel('Behaviour of spring (m)');
title('Behaviour of spring VS Time')
legend('Spring 1', 'Spring 2','Spring 3','Spring 4');
grid on

figure()
plot(z1,z2,'--r','linewidth',1.5);
xlabel('Time(s)');
ylabel('Velocity (m/s)');
title('Velocity VS Time')
legend('X direction', 'Y direction');
grid on

\end{verbatim}
\end{framed}

\begin{framed}
\begin{verbatim}
    clear all
close all
clc

n=100;
r=0.1;

load path.mat;

x=D(:,1);
y=D(:,2);


figure(1);
plot(x(1:end),y(1:end),'r','linewidth',2)
ylabel('y-axis,horizontal position')
xlabel('x-axis,vertical position')
title('Vertical position and horizontl position')
grid on;


figure(2)
plot(-3,0,'r.');
hold on 
plot(0,-3,'r.');
plot(3,0,'r.');
plot(0,3,'r.');
axis tight

for j = 1:length(x)
    
    [x1,y1] = sp1(0,-3,x(j),y(j),n,r);
    h=plot(x1,y1,'k-','LineWidth',2);
    hold on
    
    [x2,y2] = sp1(0,3,x(j),y(j),n,r);
    h=plot(x2,y2,'k-','LineWidth',2);
    
    [x3,y3] = sp1(-3,0,x(j),y(j),n,r);
    h=plot(x3,y3,'k-','LineWidth',2);
    
    [x4,y4] = sp1(3,0,x(j),y(j),n,r);
    h=plot(x4,y4,'k-','LineWidth',2);
    
    plot(-3,0,'r.');
    plot(0,-3,'r.');
    plot(3,0,'r.');
    plot(0,3,'r.');
    
    h=plot(x(j),y(j),'--rs','LineWidth',2,'MarkerEdgeColor','k',...
                'MarkerFaceColor','g',...
                'MarkerSize',30);
            grid on
	drawnow
    %pause(0.005);
	delete(h)
	hold off
end

    
\end{verbatim}
\end{framed}




























 

































































































